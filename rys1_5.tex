\documentclass[a4paper]{article}
\usepackage[pdftex,breaklinks=true]{hyperref}
\usepackage{polski}
\usepackage[utf8]{inputenc}
\renewcommand{\|}{\kern.12em}
\title{Empiryczna weryfikacja trzeciego paradoksu Otrembusa:
  ocena potencjału imaginacyjnego powiatów tatrzańskiego i~nowotarskiego}
\author{Jabcon Józef \and Voloff Sergiusz} \date{2010}
\begin{document}
\maketitle %% wstaw tytuł
%\tableofcontents % wstaw spis treści

\section {Wprowadzenie \label{punkt:wpro}}
Imagineskopia, która jest nauką o~sposobach i~technikach doraźnego 
powiększania wyobraźni przy 
szczególnym zaangażowaniu zmysłu wzroku~\cite[s.~34]{Otrembus-1977}...

\section{Ocena potencjału w~aspekcie przestrzennym}
Gęstością pendologiczną\footnote{%
Zobacz też \cite{hytz-pendologische}}.
określamy za~\cite{Otrembus-1977} wielkość 
statycznego potencjału imaginacyjnego (por.~punkt~\ref{p:sex}) na 
jednostkę powierzchni. Za~\cite{hytz-pendologische} definiujemy 
współczynnik gęstości pendolgicznej jako:
\begin{equation}  g_p = \frac{\sum_{i=1}^{N} \hat i_i}{ G' -1 } 
\end{equation}
przy czym, znaczenie symboli objaśniono w~punkcie~\ref{punkt:wpro}. 

\section{Płeć dependentów a~potencjał imaginacyjny \label{p:sex}}

Analiza danych zamieszczonych wskazuje, że:
\begin{itemize}
\item przeciętny indywidualny potencjał statyczny ($\bar \iota$) 
 jest o~ponad 18\% wyższy w~powiecie nowotarskim;
\item poziom gęstości (mierzonej w~khy/km$^2$) w~obu 
 badanych powiatach jest zbliżony.
\end{itemize}
Szukając odpowiedzi na pytanie dlaczego\dots

\begin{thebibliography}{1}
\bibitem{hytz-pendologische} Hytz J.~A., 
     \emph{Die Pendologische Methode}, Wien 1881.
\bibitem{Otrembus-1977} Otrembus-Podgrobelski~Ś.,
\emph{Wstęp do imagineskopii}, Kraków 1977, Wydawnictwo Literackie.
\end{thebibliography}
\end{document}
\end{literalexample}

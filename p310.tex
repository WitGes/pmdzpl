\documentclass[a4paper]{article}
\usepackage{array,longtable}
\usepackage{polski}
\usepackage[utf8]{inputenc}

\textheight=11\baselineskip %% smiesznie mala wysokosc kolumny

\begin{document}

\begin{longtable}{|l|>{\itshape}l|r|r|}
 \caption{Tytuł tabeli}\\\hline
 \multicolumn{4}{|c|}{Główka tabeli na pierwszej stronie}\\ \hline 
  Nazwa zwyczajowa & Nazwa łacińska & $\bar W$ (hy) & $W_{\textrm{max}}$ \\ \hline
  K1 & K2 & K3 & K4 \\ \hline
 \endfirsthead
 \hline
 \multicolumn{4}{|c|}{Główka tabeli na następnych stronach}\\
 \hline K1 & K2 & K3 & K4\\\hline
 \endhead
 \hline \multicolumn{4}{|c|}{Stopka tabeli na pierwszej stronie}\\ \hline
 \endfoot
 \hline \multicolumn{4}{|c|}{Stopka na ostatniej stronie}\\
 \hline
 \endlastfoot
Pasikonik zielony     & Tettigonia viridissima  &   2,6  &  2,9 \\
Pasikonik śpiewający  & Tettigonia cantans      &   2,7  &  3,0 \\
Karaluch              & Blatta orientalis       &   3,0  &  4,0 \\
Mucha domowa          & Musca domestica         &   4,1  &  4,2 \\
Bolimuszka kleparka   & Stomoxys calcitrans     &   0,5  &  0,5 \\
Mól włosienniczek     & Tineola bisselliella    &   0,8  &  0,9 \\
Omacnica spichrzanka  & Plodia interpunctella   &   0,8  &  0,9 \\
Rybik cukrowy         & Lepisma saccharina      &   1,1  &  1,2 \\
Mrówka faraona        & Monomorium pharaonis    &   3,5  &  3,7 \\
Pluskwa domowa        & Cimex lectularius       &   4,2  &  4,4 \\
Pstrokaczek ukraiński & Poecilimon ukrainicus   &   2,5  &  2,9 \\
 %%
Zgniłówka pokojowa    & Fannia canicularis      &   0,6  &  0,9 \\
Ścierwica mięsówka    & Sarcophaga carnaria     &   0,4  &  0,9 \\
Żyrytwa pluskwowata   & Ilyocoris cimicoides    &   0,4  &  0,9 \\
Wątlik charłaj        & Leptophyes punctatissima  & 2,0  &  2,5 \\
Trajkotka czerwona    & Psophus stridulus       &   2,2  &  2,9 \\
Zalotka spłaszczona   & Leucorrhinia caudalis   &   2,5  &  2,8 \\
Świtezianka dziewica  & Calopteryx virgo        &   6,2  &  6,1 \\
Świtezianka błyszcząca& Calopteryx splendens    &   6,3  &  6,4 \\
Husarz władca         & Anax imperator          & $-$0,1 &  0,0 \\
Skorek pospolity      & Forficula auricularia   &   0,1  &  0,2 \\
\end{longtable}


\end{document}
